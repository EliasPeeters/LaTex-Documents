\documentclass[11pt, oneside]{article}   	
\usepackage{geometry} 
\usepackage{setspace} 
\usepackage{mathpazo}              
\geometry{letterpaper} 
                  		


\usepackage{graphicx}				
		
\usepackage{amssymb}

%SetFonts

%SetFonts

\doublespacing
\setlength{\parindent}{0em} 



\title{Mathematik E-Phase Funktionen}
\author{Elias Peeters}
							

\begin{document}
\maketitle

\pagebreak


\begin{center}

\huge \textbf{Danksagung} \par


\normalsize 

\end{center}


\pagebreak

\tableofcontents
\pagebreak


\section{Funktionsuntersuchung}

\subsection{Nullestellenberechnung ganzrationaler Funktionen}

\subsubsection{Lineare Funktion}

$f(x)=3x-7$

f(x) mit Null gleichsetzen.

$0=3x-7$

$7=3x$

$x_0=\frac{7}{3}$

$(\frac{7}{3}|0)$ Schnittpunkt mit der x-Achse
\subsubsection{Quadratische Funktionen}

$g(x)=4x^2+7x-3$

g(x) mit Null gleichsetzen.

$0=4x^2+7x-3$

Um die Nullstelle einer Quadratischen Funktionzuberechen, wird die p-q-Formel angewendet. Daf\"ur darf das $x^2$ nie einen zweiten Faktor haben.

$0=x^2+\frac{7}{4}x-\frac{3}{4}$

Die Allgemeine p-q-Formel aufschreiben.

$x_{01|02}=-\frac{p}{2}\pm\sqrt{\frac{p^2}{4}-q}$

Gleichung in die p-q-Formel einsetzen.

$x_{01|02}=-\frac{8}{7}\pm\sqrt{\frac{49}{64}+\frac{3}{4}}$

$x_{01}=-\frac{3}{4}$

$x_{02}=-1$

Bemerkung: Quadratische Funktionen k�nnen eine, zwei oder keine Nullstellen haben.
\pagebreak

\subsubsection{H\"ohergradige Funktionen}


\textbf{Variante 1: Substitution}
\vspace{0,5cm}

Eine h\"ohergradige Funktion kann nur substituiert werden, wenn sie dieser Form entspricht: $f(x)=ax^{2\alpha}+bx^{\alpha}+c$

$f(x)=x^4+2x^2-1$

Substitution: $x^2=z$ => $x^4=z^2$

f(x) mit Null einsetzen und x mit z ersetzen.

$0=z^2+2z-1$

Anwenden der p-q-Formel.

$x_{01|02}=-\frac{2}{2}\pm\sqrt{\frac{4}{4}+1}$

$x_{01}=-1+\sqrt{2} \approx 0,41 $

$x_{02}=-1\sqrt{2} \approx -2,41$

R\"ucksubstitution:

$0,41=x^2$

$x_{01}=\sqrt{0,41}$

$x_{02}=-\sqrt{0,41}$


$-2,41=x^2$

$x_{03}=\sqrt{-2,41}$ (nicht m\"oglich)

$x_{04}=-\sqrt{-2,41}$ (nicht m\"oglich)


Die Funktion f(x) hat nur zwei Nullstellen, da der Inhalt einer Wurzel nicht negativ sein darf.

\pagebreak


\textbf{Variante 2: Linearfaktoren}

$f(x)=\frac{1}{2}(x+3)^3(x-1)^2(x+9)x$

Ein Produkt ist genau null, wenn einer der Faktoren null ist.

$x_{01}=0$

$x_{02}=-9$

$x_{03}=1$

$x_{04}=-3$

Man versucht, dass ein Produkt der Funktion 0 ist und damit die ganze Gleichung 0 ist.

Sind die Nullstellen einer ganz-Rationalen-Funktion bekannt, so kann man sie in Form von Linearfaktoren angeben.

\pagebreak

\subsection{Der Grad einer Funktion}

Der Grad einer Funktion ist die h\"ochste Potenz, die die Funktion hat. Bei der Funktion $f(x)=x^2$ ist der Grad 2. Bei der Funktion $g(x)=x^2+x^3$ ist der Grad drei, da zwei kleiner ist als drei. Die Funktion $f(x)=x^2*x^3$ hat jedoch den Grad  5, da die einzelnen Potenzen hier addiert werden.

















\end{document}  