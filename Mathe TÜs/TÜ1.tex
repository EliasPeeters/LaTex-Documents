\documentclass[11pt, oneside]{article}   	% use "amsart" instead of "article" for AMSLaTeX format
\usepackage{geometry}
\usepackage{setspace}                		% See geometry.pdf to learn the layout options. There are lots.
\geometry{letterpaper}                   		% ... or a4paper or a5paper or ... 
%\geometry{landscape}                		% Activate for rotated page geometry
%\usepackage[parfill]{parskip}    		% Activate to begin paragraphs with an empty line rather than an indent
\usepackage{graphicx}				% Use pdf, png, jpg, or eps§ with pdflatex; use eps in DVI mode
								% TeX will automatically convert eps --> pdf in pdflatex		
\usepackage{amssymb}
\doublespacing
\setlength{\parindent}{0em} 

%SetFonts

%SetFonts


\title{Brief Article}
\author{The Author}
%\date{}							% Activate to display a given date or no date

\begin{document}
\maketitle
\pagebreak
\section{Aufgabe 1}
Aussage 1

$f(0)=2$

Die Nullstelle der Funktion ist an dem Punkt 2. Somit kennen wir den Punkt $P_1(0|2)$ 

\vspace{0.5cm}
Aussage 2

$f(3)=11$

Die Funktion verl�uft durch den Punkt $P_1(3|11)$ 

\vspace{0.5cm}

Aussage 3

$f(3) =f(-3)$

\vspace{0.5cm}

Aussgabe 4

$f'(3)=6$

An dem Punkt x=3 ist die Tangentensteigung 6.


\section{Aufgabe 2}

Die erste Ableitung der Funktion $u(x)$ ($u'(x)$) gibt die Steigung der Tangente an dem Punkt x an.

\section{Aufgabe 3}

Die Funktion $u'(x)$ gibt die Steigung an dem Punkt x an.

\section{Aufgabe 4}

\section{Aufgabe 5}

\underline{Quadrat:} $a^2$

\underline{Rechteck:} $a*b$

\underline{Parallelogramm:} $a*b$

\underline{Trapez:} $\frac{1}{2}(a+c)*h$

\underline{Dreieck:} $\frac{1}{2}a*b$

\section{Aufgabe 6}
\begin{tabular}{|l|l|l|l|l|l|l|}
\cline{1-7}
       & -3 & -1 & 0 & 1 & 1.5 & 3 \\ \cline{1-7}
$f(x)=x^2+2$ & -7 & 1 & 2 & 3 &4.25 &11  \\ \cline{1-7}
$g(x)=\frac{3}{2}x-2$       & -6.5 & -3.5 &-2 &0.5 &0.25 & 2.5  \\ \cline{1-7}
$f(x)=\frac{1}{8}x^3+x^2+x$  & $\frac{11}{8}$ & $-\frac{1}{8}$  & 0 &$ \frac{17}{8}$& $\frac{267}{64}$ & $\frac{123}{8}$ \\ \cline{1-7}

\end{tabular}

\section{Aufgabe 7}

\underline{W\"urfel:} $a^3$

\underline{Rechteck:} $a*b*c$

\underline{Pyramide:} $\frac{1}{3}*a^2*h$

\underline{Keil:} $\frac{1}{2}*a^2*h$

\underline{Kugel:} $4\pi*r^2$

\underline{Kreiskegel:} $\frac{1}{3}\pi r^2 h$

\section{Aufgabe 9}

$m(-1)=-3$

$m(0)=0$

$m(0.5)=0.75$

Wenn man die Sekante an dem Punkt 1 bestimmen w�rde, w�rde man durch null teilen und da dies nicht m�glich ist, ist die Funktion m(x) an dem Punkt 1 undefiniert.


Die Funktion f'(1) hat den wert 3.











 



\end{document}  