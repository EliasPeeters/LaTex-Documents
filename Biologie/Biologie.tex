\documentclass[11pt, oneside]{article}   	
\usepackage{geometry}                		
\geometry{letterpaper} 
\usepackage{setspace}                  		


\usepackage{graphicx}					
\usepackage{amssymb}
\setlength{\parindent}{0em} 
\doublespacing




\title{Biologie}
\author{Elias Peeters}


\begin{document}
\maketitle
\pagebreak
\section{\"Okologie}
\subsection{Xerophyten und Hygrophyten}

Trockenpflanzen

Eine Trockenpflanzen muss mehr Wasser sparen. Bei einer Regenzeit kann sie dann 6 mal so viel Fotosynthese betreiben.

\vspace{0.5cm}

Lianen


Die Lianen brauchen dann noch mehr Kraft um das Wasser nach oben zu bef�rdern. Ausserdem wird die Pflanze instabiler.

\subsection{Das Blatt}

- Oben auf dem Blatt ist die Cuticula

- Die Cuticula ist eine Art Wax

- Damit l\"auft der Schmutz ab.

- Wenn Transpiration geschieht, verschlafen die beiden Zellen neben der Spalt\"offnung und damit verschliessen sie, da sie zusammenfallen.

\subsection{Transpirationssog}

Aufgrund osmotischer Kr\"afte entsteht ein Unterdruck, durch die Verdunstung (Transpiration) von $H_2O$ an den Bl\"attern.

\subsection{Xyem und Phloem}

Das Xylem ist das "Holzteil" $=>$ unlebend.

Das Pholem ist das "Siebteil" $=>$ lebend

\pagebreak

\subsection{Warum steigt das Wasser in der Pflanze nach oben}

Durch die Atesionkraft ziehen sich die Atome an den W�nden hoch und somit bleibt des Wasserstrahl in dem 






\end{document}  