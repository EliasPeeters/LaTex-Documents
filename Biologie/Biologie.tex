\documentclass[11pt, oneside]{article}   	
\usepackage{geometry}                		
\geometry{letterpaper} 
\usepackage{setspace}                  		


\usepackage{graphicx}					
\usepackage{amssymb}
\setlength{\parindent}{0em} 
\doublespacing




\title{Biologie}
\author{Elias Peeters}


\begin{document}
\maketitle
\pagebreak
\section{\"Okologie}
\subsection{Xerophyten und Hygrophyten}

Trockenpflanzen

Trockenpflanzen machen mehr Transpiration und stossen damit mehr Wasser aus und brauchen da\"ur mehr Spalt\"offnungen.

\vspace{0.5cm}

Lianen


Die Lianen brauchen dann noch mehr Kraft um das Wasser nach oben zu bef�rdern. Au�erdem wird die Pflanze instabiler.

\subsection{Das Blatt}

- Oben auf dem Blatt ist die Cuticula

- Die Cuticula ist eine Art Wax

- Damit l\"auft der Schmutz ab.

- Wenn Transpiration geschieht, verschlafen die beiden Zellen neben der Spalt\"offnung und damit verschliessen sie, da sie zusammenfallen.

- 





\end{document}  