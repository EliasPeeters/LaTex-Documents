\documentclass[11pt, oneside]{article}   	% use "amsart" instead of "article" for AMSLaTeX format
\usepackage{geometry} 
\usepackage{setspace}               		% See geometry.pdf to learn the layout options. There are lots.
\geometry{letterpaper}                   		% ... or a4paper or a5paper or ... 
%\geometry{landscape}                		% Activate for rotated page geometry
%\usepackage[parfill]{parskip}    		% Activate to begin paragraphs with an empty line rather than an indent
\usepackage{graphicx}				% Use pdf, png, jpg, or eps§ with pdflatex; use eps in DVI mode
								% TeX will automatically convert eps --> pdf in pdflatex		
\usepackage{amssymb}
\setlength{\parindent}{0em}
\doublespacing 

%SetFonts

%SetFonts


\title{Chemie AG}
\author{The Author}
%\date{}							% Activate to display a given date or no date

\begin{document}
\maketitle


\pagebreak

\section{Analytik}

Qualitativ: Aus welchem Stoff/ welchen Stoffen besteht die Probe?

Aus welchen Teilchen die Probe besteht?

\vspace{0.5cm}

Stoffe mit sehr eindeutigen Eigenschaften:

- Geruch nach faulen Eiern $=>$ Schwefelwasserstoff ($H_2S$)

$H_2S+2H_2O -->  2H_3O^++S^{2-}$

- Geruch nach Essig $=>$ Ethans\"aure  

- Geruch nach Ammoniak (Zersetzung von Ausscheidungen von Lebewesen) $NH_3(g)$ 

- Farbe von Kupfer (M�glichst einfach zu erkennende Beobachtung f�r einen positiven Nachweis:

\vspace{0.5cm}

\underline {Farb\"anderung}

 \"Anderung des Aggregatzustandes aus zwei klaren L�sungen entsteht ein unl�sliches Produkt $=>$ Niederschlag.
	

\vspace{0.5cm}

\underline{Flammenf\"arbung}

Wenn ein bestimmtes Teilchen (Atome oder Ionen) in eine Flamme gebracht werden, k\"onnen sie Energie aufnehmen.

Auftrennung der entstehenden Farben mit einem Prisma oder einem "feinen Gitter".


\pagebreak
\subsection{Stoffmengenkonzentration in c}

c in $\frac{mol}{L}$

$c=\frac{n}{V}$
 
 n= Stoffmenge
 
 V= Volumen
 
 $M=\frac{m}{n}$
 
 n= Stoffmenge
 
 m= Masse






















	



\end{document}  