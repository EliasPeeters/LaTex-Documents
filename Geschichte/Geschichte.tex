\documentclass[11pt, oneside]{article}   	
\usepackage{geometry} 
\usepackage{setspace} 
\usepackage{mathpazo}              
\geometry{letterpaper} 
                  		


\usepackage{graphicx}				
		
\usepackage{amssymb}

%SetFonts

%SetFonts

\doublespacing
\setlength{\parindent}{0em} 



\title{Geschichte E-Phase}
\author{Elias Peeters}
							

\begin{document}
\maketitle

\pagebreak

\tableofcontents

\pagebreak

\section{Imperialismus}

Interpretation

Die vorliegende Quelle ist ein politisches Testament aus den Jahre 1877 und wurde von dem britischen Kolonialpolitiker Cecil Rhodes, in einem Alter von 27 Jahren verfasst. In der Quelle nennt Rhodes den Grund f\"ur die milit�rische Aufr\"ustung, welche 1880 in vielen britischen Kolonien durchgef\"uhrt wurde.

In dem Testament sagt Cecil Rhodes, dass es nur eine wahre Rasse g\"abe, die britische, und das es deshalb viel besser w\"are, wenn die ganze Welt dieser Rasse angeh\"oren w\"urde. Am Ende der Quelle sagt Rhodes auch nochmal, dass dies nicht nur ein Traum sei, sondern auch m\"oglich sei. Dies verdeutlicht, dass die Rassen Einteilung von ihm nicht nur so gesagt wurde, sondern auch ernst gemeint war.

\pagebreak
\subsection{Richtig Interpretieren}

Formale Kennzeichen

- Wer hat die Karikatur geschaffen oder in Auftrag gegeben?

- Wann und wo ist sie entstanden bzw. ver\"offentlicht worden?

\vspace{0.5cm}

Bildinhalt

- Wen oder was zeigt die Karikatur

- Was wird thematisiert?

- Welche Darstellungsmittel werden verwendet und was bedeuten sie?

\vspace{0.5cm}

Historischer Kontext

- Auf welches Ereignis, welchen Sachverhalt oder welche Person bezieht sich die Karikatur?

- Auf welche politische Diskussion spielt sie an?

- Wozu nimmt der Karikaturist konkret Stellung?

\vspace{0.5cm}

Intention

- An welche Adressaten wendet sich die Karikatur?

- Welchen Standpunkt nimmt der Karikaturist ein?

- Welche Aussageabsicht verfolgt er?

- Inwiefern unterst�tzt ein eventueller Text die Wirkung der Zeichnung?

- Welche Wirkung wollte der Karikaturist beim zeitgen�ssischen Betrachter erzielen?

\vspace{0.5cm}

Bewertung und Fazit

- Wie l�sst sich die Aussage der Karikatur insgesamt einordnen und bewerten?

- Wurde das Thema aus heutiger Sicht sinnvoll und �berzeugend Gestaltet?

- Welche Auffassung vertreten Sie zu de Karikatur?

\pagebreak

\subsection{Interpreation: "Pl\"underer der Welt"}

Formale Kennzeichen

- Vom deutsch-Amerikaner Thomas Nast

- Im US-Magazin Harper's Weekly vom 20. Juli 1885

\vspace{0.5cm}

Bildinhalt

- Bismarck links, John Bull in der Mitte und Zar Alexander III rechts stehen vor einem Globus der Afrika, Europa und Asien zeigt

- John Bull: national Personifikation (Nationalallegorie: Die Eigenschaften eines Landes in einer Person vertreten)

- Zar Alexander III: 1881 bis 1894 Kaiser von Russland.

- Jeder stellt sein Land dar

- Der Deutsche nimmt L\"ander aus Afrika

- Der Russe nimmt L\"ander aus Asien

- Der Brite nimmt gar kein Land

- Die S\"acke des Deutschen und des Russen sind noch relativ leer

- Der Sack der Briten ist bereits gut gef\"ullt.

- Der Brite schaut nur noch zu

\vspace{0.5cm}

Historischer Kontext

- Die Karikatur ziegt die Zeit des Imperialismus

- alle Gro�m\"achte Versuchen so viele wie m�gliche noch nicht in Besitz anderer M\"achte an sich greifen

- Der Karikaturist sagt damit, dass es f�r die M\"achte einfach war eine Kolonie zu erobern (nur das einpacken) und die Kolonie an sich nichts Wert war. Sie war nur ein Symbol von Macht gegen\"uber den anderen. 

- Die S\"acke stellen dar, dass die Einzelnen L\"ander gierig waren

 












\end{document}  