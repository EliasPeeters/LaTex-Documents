\documentclass[11pt, oneside]{article}   	% use "amsart" instead of "article" for AMSLaTeX format
\usepackage{geometry} 
\usepackage{setspace}               		% See geometry.pdf to learn the layout options. There are lots.
\geometry{letterpaper}                   		% ... or a4paper or a5paper or ... 
%\geometry{landscape}                		% Activate for rotated page geometry
%\usepackage[parfill]{parskip}    		% Activate to begin paragraphs with an empty line rather than an indent
\usepackage{graphicx}				% Use pdf, png, jpg, or eps§ with pdflatex; use eps in DVI mode
								% TeX will automatically convert eps --> pdf in pdflatex		
\usepackage{amssymb}
\doublespacing
\setlength{\parindent}{0em} 
%SetFonts

%SetFonts


\title{Religion}
\author{The Author}
%\date{}							% Activate to display a given date or no date

\begin{document}
\maketitle
\pagebreak

\section{Arbeit mit der Bibel}
\subsection{Der historische Jesus von Nazareth}

Quellen f\"ur die historische R\"uckfrag nach Jesus

\vspace{0.5cm}

- Die biblischen Quellen (Evangelien: Lk, Mk, Mt, Joh)

- Die ausserbiblische Quellen chr. Quellen (Thomas-Evangelium,...)

- Die nicht christlichen Quellen

- Die arch\"aologischen Quellen (M\"unzen, Inschriften, Vasen...)

$=>$ Jesus Christus hat es historisch gegeben


\pagebreak


\subsection{Die synoptischen Evangelien}

1. Erkl\"are die 2 Quellen Theorie.

Die zwei Quellen Theorie besagt, dass Markus, welches das \"altestes Evangelium ist, die Quelle f\"ur Matth\"aus und Lukas war. Diese beiden Evangelien verwenden aber bei als weitere Quellen, eine weitere gemeinsame Quelle. Diese bestand wahrscheinlich aus Ausspr\"uchen Jesu.

\vspace{0.5cm}

2. Sind die Evangelien verl\"asslich

Alle Evangelien bestehen nur aus Erz\"ahlungen und wurden erst Jahre sp\"ater niedergeschrieben. Dabei gehen nat\"urlich viele Informationen verloren und es werden weitere dazugedichtet. Eine Verl\"asslichkeit kann ein Evangelium nie geben. 


\pagebreak







\end{document}  